

% UsersGuideGrOptics
% Charlie Duke, Grinnell College

\documentclass{article}
\usepackage{latexsym,amssymb,amsmath,graphics}

\begin{document}

\title{GrOptics User's Guide \\ Version 2.2}
\author{Charlie Duke \\
Grinnell College \\
Grinnell, Iowa 50112 \\
duke@grinnell.edu
\and
Akira Okumura \\
Solar-Terrestrial Environment Laboratory 
\\Nagoya University \\
Furo-cho, Chikusa-ku, Nagoya 464-8601, Japan \\
oxon@mac.com}

\date{July 15, 2012}

\maketitle
\begin{abstract}

GrOptics is a detailed simulation program for ray-tracing Cherenkov photons
through large arrays of atmospheric Cherenkov telescopes. Shower packages, such
as GrISU \cite{GRISU} and CORSIKA \cite{CORSIKA}, provide Cherenkov 
photons after conversion to GrISU format. 
The output to a ROOT file records
individual photons striking the camera surface.  The package models
both VERITAS Davies-Cotton (DC) and Schwarzchild-Coudee SC) 
telescopes with all
telescope parameters taken from input files. There is no limit to the number
or type of array telescopes. Adding new telescope types, input and
output formats, etc. is 
straightforward using standard C++ coding techniques with existing base
classes. Reference \cite{GROPTICS} gives the code download site.
       
\end{abstract}
%-------------------------------------------------------------
%-------------------------------------------------------------
\section{Introduction}\label{S:INTRO}
GrOptics is a detailed C++ Monte Carlo ray-tracing program to 
study the passage of 
atmospheric Cherenkov photons through telescopes designed to study 
atmospheric Cherenkov air showers.  
Photons produced by standard air shower 
codes enter the telescope; the output 
ROOT file contains tree records of the photons that strike the telescope
cameras.  GrOptics provides the input to the CARE  telescope 
electronics code \cite{CARE}.

There are no intrinsic limits to the number or type of telescopes placed in 
the air Cherenkov telescope (ACT) array. Currently, the code contains two 
concrete telescope 
classes: for VERITAS DC telescopes and for SC 
telescopes. All array and telescope parameters are placed in input pilot or
configuration files.   

\subsection{Installation}\label{SS:INSTALL}

\begin{enumerate}    
\item
  GrOptics relies heavily on ROOT\cite{ROOT}.  I use the following
  installation method:
  \begin{itemize} 
    \item Be sure you have installed gsl for use by ROOT
    \item Download the ROOT source package \cite{ROOT}
    \item Follow package instructions to configure and to make, but 
      do not specify a --prefix directory with configure.
    \item Setup all necessary ROOT environmental variables 
      by sourcing \emph{thisroot.sh} or \emph{thisroot.csh} in your 
      \textless rootDirectory\textgreater/bin
  \end{itemize}
  \item 
  The SC telescope class uses the ROBAST 
  package \cite{ROBAST}. 
  ROBAST (ROot BAsed Simulator for ray Tracing) is a non-sequential 
  ray tracing program which utilizes the 3D geometry library in ROOT. 
  The ROBAST package is automatically downloaded
  by \emph{curl} within \emph{make} when producing the GrOptics executable.
 
\item
  After installing
  ROOT, download the GrOptics git repository \cite{GROPTICS}.  
  Go to the GrOptics directory and run \emph{make} to produce the 
  grOptics executable.

\item
  To test the installation, 
  execute \emph{grOptics} from the GrOptics directory. The code  will use the
  default configuration and pilot files and a test Cherenkov 
  photon file, all within the GrOptics/Config directory, 
  to produce an output root file, \emph{photonLocation.root}. Other test 
  possibilities (see later sections) for using this configuration are 
  telescope drawings and psf camera plots.

\end{enumerate}


\subsection{QuickStart}\label{SS:QSTART}

  The \emph{grOptics} executable can use the
  configuration and pilot files and a test Cherenkov 
  photon file, all within the GrOptics/Config directory, downloaded with 
  the git repository.  The file,
  \emph{opticsSimulation.pilot} steers the simulation and specifies the 
  output. The file, \emph{arrayConfig.cfg}, defines the ACT array. You'll see
  the downloaded file defines a four-telescope DC array that is
  compatible with the \emph{photon.cph} test input Cherenkov photon file.  
  The parameters in these files are documented both in these files and
  in following sections in this document.

\begin{enumerate} 
\item
  Execute the grOptics code from the GrOptics directory to produce 
  \emph{photonLocation.root} as specified in \emph{opticsSimulation.pilot}.
  You'll find the output trees with records of the photons on the camera
  surface for each telescope in this file. You'll also find a history ROOT
  file for each telescope which documents the history of each photon incident
  onto the telescope (useful for debugging). You can turn off the creation 
  of these files by removing the leading asterisk of the \emph{PHOTONHIST}

\item
  To create an \emph{opengl} drawing of a Davies-Cotton telescope 
  using the ROOT geometry classes,
  add a leading asterisk to the 
  \emph{DRAWTEL} record in \emph{opticsSimulation.pilot} and execute
  \emph{grOptics}. Note that for
  DC telescopes, the code does not draw the mirror facets as
  the ray-tracing for the facets does not use the ROOT geometry classes.
  Have a look at the \emph{canSpot} figure.

\item
  To replace an DC telescope with an SC telescope, open the 
  \emph{arrayConfig.cfg} file and activate the 
  \emph{TELFAC SC} SC telescope factory record. Then, in one of the 
  \emph{ARRAYTEL} records replace the \emph{DC} with \emph{SC}. You can then
  run the same tests as described above. Note that the ROOT geometry classes
  will draw the complete SC telescope (in contrast to the DC telescope
  drawing)

\item
  To produce a series of spot patterns on the camera, activate the 
  \emph{TESTTEL} record in \emph{opticsSimulation.pilot} by adding an asterisk.
  Since the code produces the spot photons internally, change the \emph{NSHOWER}
  from $-1$ and $-1$ to $1$ and $1$ (fixing this requirement 
  is on my to do list). Execute \emph{grOptics}.
\end{enumerate}
%-------------------------------------------------------------
%-------------------------------------------------------------
\section{Code Overview}\label{S:OVERV}

See \emph{DevelopersGuideGrOptics.pdf} for more details. This section
only introduces standard telescopes, telescope arrays,
and the various coordinate systems used in GrOptics.

GrOptics uses standard C++ plus ROOT.
 
\subsection{Standard Telescopes}\label{SSS:STEL} 

  The GrOptics package currently produces telescopes from each of two
  telescope factories, one for DC telescopes and one for SC telescopes.
  All telescope parameters are in configuration files to maintain maximum
  flexibility.  However, having separate configuration files for each array
  telescope leads to large, unmanagable file sizes. Thus, the factories 
  can produce a limited number of standard telescopes based on configuration
  files in the \emph{GrOptics/Config} directory.  The factories then use
  edit records taken from the configuration files to change specific 
  telescope parameters, e.g. the mirror reflectivity, thus maintaining
  complete flexibility for parameter selection for individual telescopes. 
  The edit
  records use matlab colon notation so that a single record may change a 
  parameter for a range of telescopes and elements within each
  telescope, e.g., facet reflectivities.
  Edit records for additional parameters can easily be added.

  None of the telescope dimensions or super structures may be edited.  
  Changing these dimensions requires a separate standard telescope. 

\subsection{Telescope Arrays}

  The telescope factories provide individual standard telescopes, after
  editing,  to the ACT array
  as defined by telescope type, standard ID number, ground location, and
  pointing offset. Telescope numbering starts at $1$, not $0$.

\subsection{Coordinate Systems}

Understanding the coordinate systems used in our simulation codes 
is always difficult, especially if your memory is as bad as mine.  
So, these descriptions should help.

\subsubsection{GrISU Ground Coordinate System}
  The GrISU Cherenkov files either from GrISU or from the 
CORSIKA I/O Reader use the following coordinate system:

\noindent
\hspace*{20pt}x:  East \\ 
\hspace*{20pt}y:  South \\
\hspace*{20pt}z:  Down \\
to form a right-handed system

\subsubsection{GrOptics ground coordinate system}

The incoming GrISU produced incoming photons are immediately transformed
by GrOptics to the system:

\noindent
\hspace*{20pt}x: East \\
\hspace*{20pt}y: North\\
\hspace*{20pt}z: up\\
forming a right-handed system. Note that the telescope locations in the
VERITAS GrISU configuration file use this coordinate system.
 
\subsubsection{Telescope Coordinate System}  
  Each telescope has a coordinate system with origin at the 
  telescope rotation 
  point and with z axis pointed toward the sky along the optic axis of the 
  telescope.  In stow position (DC telescopes), the z axis is horizontal and 
  pointed toward the North,  
  the y axis is down, and 
  the x axis is toward the East.  To point the telescope to a given location on the 
  sky, GrOptics first rotates the telescope about the vertical through the azimuthal angle. 
  Then, it raises the telescope about its new x axis through the elevation angle.  This
  new coordinate system is the "telescope coordinate system".  Prior to injection 
  into the GTelescope class, the photons undergo a coordinate transformation to this
  telescope coordinate system.

\subsubsection{Array Coordinate System} 
  Prior to added pointing misalignments, all telescopes point to the same location
  on the sky with coordinate systems previously defined.  
  The array coordinate system is similarly defined, but its
  origin is at the origin of the array.

\subsubsection{Camera Coordinate System}
  For historical reasons, the camera coordinate system's y axis is a reflection of the 
  y-axis of the telescope coordinate system. For the VERITAS telescope, stand in front of the
  camera with the telescope in stow position. The camera y axis is up and the camera x axis
  is to your right (to the East).  The telescope y-axis is down; its x axis is to the right.
  
  In the output tree of grOptics, the photon locations on the camera are in camera coordinates,
  both for DC and for SC telescopes.  I'll add an input flag later to select photon camera
  locations in either telescope or camera coordinates.

%-------------------------------------------------------------
%-------------------------------------------------------------
\section{Input Files}\label{S:INPUT}

%-------------------------------------------------------------
\subsection{Array Configuration}\label{SS:ARRAY}

%-------------------------------------------------------------
\subsection{Standard Telescopes}\label{SSS:STEL1}


%-------------------------------------------------------------
\subsubsection{Davies-Cotton Telescopes}\label{SSS:DCTEL}


%-------------------------------------------------------------
\subsubsection{Schwarzchild-Coudee Telescopes}\label{SSS:SCTEL}

%-------------------------------------------------------------
%-------------------------------------------------------------
\section{Data Files}\label{S:DATAF}
%-------------------------------------------------------------
\subsection{Input Data Files}\label{SS:INDATA}


%-------------------------------------------------------------
\subsection {Output Data Files}\label{SS:ODATA}

%-------------------------------------------------------------
%-------------------------------------------------------------
\section{Graphical Output Options}\label{S:GRAPH}

%-------------------------------------------------------------
%-------------------------------------------------------------
\begin{thebibliography}{4}
  \bibitem{GROPTICS}
    GrOptics git repository (read only) \\
    git clone http://gtlib.gatech.edu/pub/IACT/GrOptics.git

  \bibitem{GRISU}
    GrISU download site \\
    http://www.physics.utah.edu/gammaray/GrISU/

  \bibitem{CORSIKA}    
    CORSIKA: A Monte Carlo Code to Simulate Extensive Air Showers\\
    D. Heck, J. Knapp, J.N. Capdevielle, G. Schatz, T. Thouw \\
    Forschungszentrum Karlsruhe Report FZKA 6019 (1998)

 \bibitem{CARE}
   CARE git repository (read only) \\
   git clone http://gtlib.gatech.edu/pub/IACT/CARE.git

 \bibitem{ROOT} 
   Rene Brun and Fons Rademakers,\\
   ROOT - An Object Oriented Data Analysis Framework,\\
   Proceedings AIHENP'96 Workshop, Lausanne, Sep. 1996, \\
   Nucl. Inst. \& Meth. in Phys. Res. A 389 (1997) 81-86. \\
   See also http://root.cern.ch/drupal/

 \bibitem{ROBAST} 
   Development of Non-sequential Ray-tracing Software for Cosmic-ray Telescopes
   Authors: Akira Okumura, Masaaki Hayashida, Hideaki Katagiri, Takayuki Saito,
   Vladimir Vassiliev.  http://arxiv.org/abs/1110.4448
   Download site http://sourceforge.net/projects/robast/

\end{thebibliography}


\end{document}
