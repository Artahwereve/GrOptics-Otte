% UsersGuideGrOptics
% Charlie Duke, Grinnell College

\documentclass{article}
\usepackage{latexsym,amssymb,amsmath,graphics}

\begin{document}

\title{GrOptics User's Guide \\ Version 2.2}
\author{Charlie Duke \\
Grinnell College \\
Grinnell, Iowa }

\date{July 15, 2012}

\maketitle
\begin{abstract}

GrOptics is a detailed simulation program for ray-tracing Cherenkov photons
through large arrays of atmospheric Cherenkov telescopes. Shower packages, such
as GrISU \cite{GRISU} and CORSIKA \cite{CORSIKA}, provide Cherenkov 
photons after conversion to GrISU format. 
The output to a ROOT file records
individual photons striking the camera surface.  The package models
both VERITAS Davies-Cotton and Schwarzchild-Coudee telescopes with all
telescope parameters taken from input files. There is no limit to the number
or type of array telescopes. Adding new telescope types, input and
output formats, etc. is 
straightforward using standard C++ coding techniques with existing base
classes. Reference \cite{GROPTICS} gives the code download site.
       
\end{abstract}
%-------------------------------------------------------------
%-------------------------------------------------------------
\section{Introduction}\label{S:INTRO}
GrOptics is a detailed C++ Monte Carlo ray-tracing program to 
study the passage of 
atmospheric Cherenkov photons through telescopes designed to study 
atmospheric Cherenkov air showers.  
Photons produced by standard air shower 
codes enter the telescope; the output 
ROOT file contains tree records of the photons that strike the telescope
cameras.  GrOptics provides the input to the CARE  telescope 
electronics code \cite{CARE}.

There are no intrinsic limits to the number or type of telescopes placed in 
the air Cherenkov telescope (ACT) array. Currently, the code contains two 
concrete telescope 
classes: for VERITAS Davies-Cotton telescopes and for Schwarzchild-Coudee 
telescopes. All array and telescope parameters are placed in input pilot or
configuration files.   

\subsection{Installation}\label{SS:INSTALL}

\begin{enumerate}
\item
    
\item
  GrOptics relies heavily on ROOT \cite{ROOT}.  I use the following
  installation method:
  \begin{itemize} 
    \item Download the ROOT source package \cite{ROOT}
    \item Follow package instructions to configure and to make
    \item Setup ROOT environmental variables by sourcing \emph{thisroot.sh}
      or \emph{thisroot.csh} in \textless rootDirectory\textgreater/bin
  \end{itemize}
  \item 
  The Schwarzchild-Coudee telescope class uses the ROBAST 
  package \cite{ROBAST}. 
  ROBAST (ROot BAsed Simulator for ray Tracing) is a non-sequential 
  ray tracing program which utilizes the 3D geometry library in ROOT. 
  Its function is very simple compared to Geant4 Optical, 
  but much easier to use. The ROBAST package is automatically downloaded
  by \emph{make} when producing the GrOptics executable.
 
\item
  Download the GrOptics git repository \cite{GROPTICS}. After installing
  ROOT, go to the GrOptics directory and run \emph{make} to produce the 
  grOptics executable.

\item
  Executing \emph{grOptics} from the GrOptics directory will use the
  default configuration and pilot files and a test Cherenkov photon file
  to produce an output root file, photonLocation.root. Other output 
  possibilities (see later sections) for this test configuration are 
  telescope drawings and psf camera plots.

\end{enumerate}


\subsection{QuickStart}\label{SS:QSTART}

%-------------------------------------------------------------
%-------------------------------------------------------------
\section{Code Overview}\label{S:OVERV}

%-------------------------------------------------------------
%-------------------------------------------------------------
\section{Input Files}\label{S:INPUT}

%-------------------------------------------------------------
\subsection{Array Configuration}\label{SS:ARRAY}

%-------------------------------------------------------------
\subsection{Standard Telescopes}\label{SSS:}


%-------------------------------------------------------------
\subsubsection{Davies-Cotton Telescopes}\label{SSS:}


%-------------------------------------------------------------
\subsubsection{Schwarzchild-Coudee Telescopes}\label{SSS:}

%-------------------------------------------------------------
%-------------------------------------------------------------
\section{Data Files}\label{S:DATAF}
%-------------------------------------------------------------
\subsection{Input Data Files}\label{SS:INDATA}


%-------------------------------------------------------------
\subsection {Output Data Files}\label{SS:ODATA}

%-------------------------------------------------------------
%-------------------------------------------------------------
\section{Graphical Output Options}\label{S:GRAPH}

%-------------------------------------------------------------
%-------------------------------------------------------------
\begin{thebibliography}{4}
  \bibitem{GROPTICS}
    GrOptics git repository (read only) \\
    git clone http://gtlib.gatech.edu/pub/IACT/GrOptics.git

  \bibitem{GRISU}
    GrISU download site \\
    http://www.physics.utah.edu/gammaray/GrISU/

  \bibitem{CORSIKA}    
    CORSIKA: A Monte Carlo Code to Simulate Extensive Air Showers\\
    D. Heck, J. Knapp, J.N. Capdevielle, G. Schatz, T. Thouw \\
    Forschungszentrum Karlsruhe Report FZKA 6019 (1998)

 \bibitem{CARE}
   CARE git repository (read only) \\
   git clone http://gtlib.gatech.edu/pub/IACT/CARE.git

 \bibitem{ROOT} 
   Rene Brun and Fons Rademakers,\\
   ROOT - An Object Oriented Data Analysis Framework,\\
   Proceedings AIHENP'96 Workshop, Lausanne, Sep. 1996, \\
   Nucl. Inst. \& Meth. in Phys. Res. A 389 (1997) 81-86. \\
   See also http://root.cern.ch/drupal/

 \bibitem{ROBAST} 
   Development of Non-sequential Ray-tracing Software for Cosmic-ray Telescopes
   Authors: Akira Okumura, Masaaki Hayashida, Hideaki Katagiri, Takayuki Saito,
   Vladimir Vassiliev.  http://arxiv.org/abs/1110.4448
   Download site http://sourceforge.net/projects/robast/

\end{thebibliography}


\end{document}
